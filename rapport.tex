% Dépendences : 
% pip install pygments
% Python3

% Installation de pip : (Windows)
% wget https://bootstrap.pypa.io/get-pip.py
% python3 get-pip.py
% Rajout dans le PATH, cf. message

% Compilation : pdflatex --shell-escape rapport.tex

\documentclass[review, 1p]{elsarticle}
\usepackage[T1]{fontenc}
\usepackage[utf8]{inputenc}
\usepackage{csquotes}
\usepackage[french]{babel}
\usepackage{fancyhdr}
\usepackage{titlesec} % titlespacing %
\usepackage{listings} % lstnewenvironment %
\usepackage{textcomp}
\usepackage{regexpatch}
\usepackage[usenames,dvipsnames,svgnames,table]{xcolor}
\usepackage{parskip}
\usepackage{graphicx}
\usepackage{pdfpages}
\usepackage{caption}
\usepackage{titletoc}
\usepackage[table]{xcolor}
\usepackage{minted}
\usepackage[]{algorithm2e}
\usepackage{amsmath,amssymb,amsfonts}
\usepackage{graphicx}
\usepackage{textcomp}
\usepackage{xcolor}
\usepackage{hyperref}
\hypersetup{
  colorlinks,
  citecolor=black,
  filecolor=black,
  linkcolor=black,
  urlcolor=black
}
\newtheorem{thm}{Théorèle}
\newtheorem{lem}[thm]{Lemme}
\newdefinition{rmk}{Remarque}
\newproof{pf}{Preuve}

\newcounter{linecounter}
\newcommand{\linenumbering}{\ifthenelse{\value{linecounter}<10}{(0\arabic{linecounter})}{(\arabic{linecounter})}}
\renewcommand{\line}[1]{\refstepcounter{linecounter}\label{#1}\linenumbering}
\newcommand{\resetline}[1]{\setcounter{linecounter}{0}#1}
\renewcommand{\thelinecounter}{\ifnum \value{linecounter} > 9\else 0\fi \arabic{linecounter}}

% Enlever le footer spécifique à Elsevier

\keywordtitle{Mots-clés\,}
\abstracttitle{Résumé}

\makeatletter
\regexpatchcmd*{\@makecaption}{:}{\cA:}{}{}
\regexpatchcmd*{\keyword}{:}{\cA:}{}{}
\def\ps@pprintTitle{%
 \let\@oddhead\@empty
 \let\@evenhead\@empty
 \def\@oddfoot{}%
 \let\@evenfoot\@oddfoot}
% fix english
\xpatchcmd{\printFirstPageNotes}
  {Email addresses}
  {Adresses email}{}{}
\xpatchcmd{\printFirstPageNotes}
  {Email address}
  {Adresse email}{}{}
\regexpatchcmd*{\printFirstPageNotes}{:}{\cA:}{}{}
\makeatother

\begin{document}
\begin{frontmatter}
    \title{Méthode de Newton et de Quasi-Newton BFGS (Algorithme de BRoyden, Fletcher, Goldfarb et Shanno)}
    \begin{center}
        \includegraphics[width=0.5\textwidth]{logo_UTC.jpg}
    \end{center}
    \author{Mathilde Rineau}
    \author{Mathilde Le Moel}
    \author{Pascal Quach}
    \author{Félix Poullet-Pagès}

    \begin{abstract}
        La méthode de Newton est une méthode itérative de recherche d'un zéro du second ordre. Elle se repose sur la méthode du point fixe.
    \end{abstract}
\end{frontmatter}
\tableofcontents

\section*{Introduction}
Une intro

\begin{algorithm*}[ht]
    \centering{
        \fbox{
            \begin{minipage}[t]{150mm}
                \footnotesize
                \renewcommand{\baselinestretch}{2.5}
                \resetline
                \begin{tabbing}
                    aaaA\=aaaA\=aaA\=aaA\=aaA\=aaA\kill

                    {\bf  Initialisation:}
                    \\
                    \line{P2-01} \> {\bf Si} 1ère ligne
                    \\
                    \line{P2-02} \>\> 2ème ligne si condition en \ref{P2-01}
                    %------------------


                \end{tabbing}
                \normalsize
            \end{minipage}
        }
        \medbreak
        \caption{Le titre}
        \label{fig:label_algo}
    }
\end{algorithm*}



\end{document}
