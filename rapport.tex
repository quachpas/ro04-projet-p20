% Dépendences : 
% pip install pygments
% Python3

% Installation de pip : (Windows)
% wget https://bootstrap.pypa.io/get-pip.py
% python3 get-pip.py
% Rajout dans le PATH, cf. message

% Compilation : pdflatex --shell-escape rapport.tex

% Citations de sources bibliographiques
% http://merkel.texture.rocks/Latex/natbib.php?lang=fr

% Documentation classe 
% https://www.elsevier.com/__data/assets/pdf_file/0009/56844/elsdoc2.pdf
\documentclass[1p]{elsarticle}
\usepackage[T1]{fontenc}
\usepackage[utf8]{inputenc}
\usepackage{csquotes}
\usepackage[french]{babel}
\usepackage{fancyhdr}
\usepackage{titlesec} % titlespacing %
\usepackage{listings} % lstnewenvironment %
\usepackage{textcomp}
\usepackage{regexpatch}
\usepackage[usenames,dvipsnames,svgnames,table]{xcolor}
\usepackage{parskip}
\usepackage{graphicx}
\usepackage{pdfpages}
\usepackage{caption}
\usepackage{titletoc}
\usepackage[table]{xcolor}
\usepackage{minted}
\usepackage[french]{algorithm2e}% http://tug.ctan.org/macros/latex/contrib/algorithm2e/doc/algorithm2e.pdf
\usepackage{amsmath,amssymb,amsfonts}
\usepackage{graphicx}
\usepackage{textcomp}
\usepackage{xcolor}
\usepackage{hyperref}
\hypersetup{
  colorlinks,
  citecolor=black,
  filecolor=black,
  linkcolor=black,
  urlcolor=black
}
\newtheorem{thm}{Théorèle}
\newtheorem{lem}[thm]{Lemme}
\newdefinition{rmk}{Remarque}
\newproof{pf}{Preuve}

\newcounter{linecounter}
\newcommand{\linenumbering}{\ifthenelse{\value{linecounter}<10}{(0\arabic{linecounter})}{(\arabic{linecounter})}}
\renewcommand{\line}[1]{\refstepcounter{linecounter}\label{#1}\linenumbering}
\newcommand{\resetline}[1]{\setcounter{linecounter}{0}#1}
\renewcommand{\thelinecounter}{\ifnum \value{linecounter} > 9\else 0\fi \arabic{linecounter}}

% Enlever le footer spécifique à Elsevier

\keywordtitle{Mots-clés\,}
\abstracttitle{Résumé}

\makeatletter
\regexpatchcmd*{\@makecaption}{:}{\cA:}{}{}
\regexpatchcmd*{\keyword}{:}{\cA:}{}{}
\def\ps@pprintTitle{%
 \let\@oddhead\@empty
 \let\@evenhead\@empty
 \def\@oddfoot{}%
 \let\@evenfoot\@oddfoot}
% fix english
\xpatchcmd{\printFirstPageNotes}
  {Email addresses}
  {Adresses email}{}{}
\xpatchcmd{\printFirstPageNotes}
  {Email address}
  {Adresse email}{}{}
\regexpatchcmd*{\printFirstPageNotes}{:}{\cA:}{}{}
\makeatother

\begin{document}
\begin{frontmatter}
    \title{Méthode de Newton et de Quasi-Newton BFGS (Algorithme de BRoyden, Fletcher, Goldfarb et Shanno)}
    \begin{center}
        \includegraphics[width=0.5\textwidth]{logo_UTC.jpg}
    \end{center}
    \author{Mathilde Rineau}
    \author{Mathilde Le Moel}
    \author{Pascal Quach}
    \author{Félix Poullet-Pagès}

    \begin{abstract}
        La méthode de Newton est une méthode itérative de recherche d'un zéro d'une fonction $g$, qui se repose sur la méthode du point fixe. Elle requiert cependant le calcul coûteux de la dérivée ou matrice jacobienne de $g$ dans le cas d'une fonction à plusieurs variables, et également son inversion. On parle de méthodes de quasi-Newton lorsque la matrice jacobienne - et généralement son inverse - sont remplacées par une approximation. Appliquée à un problème d'optimisation, où l'on cherche l'optimum d'une fonction $f$, la méthode de Newton force le calcul de la matrice hessienne de $f$, car on cherche les zéros du gradient de $f$. La méthode de Broyden-Fletcher-Goldfarb-Shanno (BFGS) est une méthode quasi-Newton qui se repose sur l'approximation de la matrice hessienne par analyse des gradients successifs. La matrice hessienne n'est pas calculée à chaque itération de la méthode, mais mise à jour itérativement en prenant une estimation de la matrice hessienne initiale.
    \end{abstract}
\end{frontmatter}
\clearpage
\tableofcontents
\clearpage
\begin{algorithm*}[ht]
    \centering{
        \fbox{
            \begin{minipage}[t]{150mm}
                \footnotesize
                \renewcommand{\baselinestretch}{2.5}
                \resetline
                \begin{tabbing}
                    aaaA\=aaaA\=aaA\=aaA\=aaA\=aaA\kill

                    {\bf  Initialisation:}
                    \\
                    \line{P2-01} \> {\bf Si} 1ère ligne
                    \\
                    \line{P2-02} \>\> 2ème ligne si condition en \ref{P2-01}
                    %------------------
                \end{tabbing}
                \normalsize
            \end{minipage}
        }
        \medbreak
        \caption{Exemple d'algorithme}
        \label{fig:label_algo}
    }
\end{algorithm*}
\section{Méthode de Newton}
\subsection{Origines}
\subsection{Principe}
\subsection{Résultats connus}
\subsection{Qualités et faiblesses}
\section{Méthode de BFGS}
\subsection{Origines}
\subsection{Principe}
\subsection{Résultats connus}
\subsection{Qualités et faiblesses}
\section{Implémentation sur SCILAB}
\subsection{Recherche linéaire : condition de Wolfe}
\subsubsection{Algorithme}
\subsubsection{Programme}
\subsubsection{Conditions d'arrêts et précision}
\subsubsection{Difficultées techniques}
\subsection{Méthode de Newton}
\subsubsection{Algorithme}
\subsubsection{Programme}
\subsubsection{Conditions d'arrêts et précision}
\subsubsection{Difficultées techniques}
\subsection{Méthode de BFGS}
\subsubsection{Algorithme}
\subsubsection{Programme}
\subsubsection{Conditions d'arrêts et précision}
\subsubsection{Difficultées techniques}
\subsection{Affichage des résultats}
\subsubsection{Indicateurs possibles}
\subsection{Exemples d'optimisation}
\subsubsection{Méthode de Newton}
\subsubsection{Exemple convergent}
\subsubsection{Exemple divergent}
\subsubsection{Exemple de convergence difficile}
\subsubsection{Méthode de BFGS}
\subsubsection{Exemple convergent}
\subsubsection{Exemple divergent}
\subsubsection{Exemple de convergence difficile}
\subsection{Critiques et alternatives}
\subsubsection{Critiques}
% \paragraph{Prise en main du langage}
% \paragraph{Temps de calcul}
\subsubsection{Alternatives}
\section{Applications}
\subsection{Problème 1 : ...}
% \paragraph{Indicateur 1 : courbe de convergence (exemple)}



% Bibliographie
% http://merkel.texture.rocks/Latex/natbib.php?lang=fr
\bibliographystyle{elsarticle-num}
\bibliography{bibliographie}


\end{document}
